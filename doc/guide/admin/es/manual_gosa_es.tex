\documentclass[a4paper,spanish,10pt]{book}

\usepackage[latin1]{inputenc}
\usepackage[T1]{fontenc}
\usepackage[pdftex]{graphicx}
\usepackage[spanish]{babel}
\usepackage{times}
\usepackage{multirow}
\usepackage[latex2html,backref,pdftitle={GOsa}]{hyperref}
\usepackage{html,color}
\usepackage{latexsym}
\usepackage{anysize}
\usepackage{float}
\usepackage{longtable}
\usepackage{verbatim}

\author{Alejandro Escanero Blanco}
\title{Administraci�n de Sistemas con GOsa }
\date{28-03-05}

\marginsize{3cm}{2cm}{2.5cm}{2.5cm}

\begin{document}
\renewcommand{\contentsname}{�ndice General}
\renewcommand{\listfigurename}{Lista de Figuras} 
\renewcommand{\listtablename}{Lista de Tablas}
\renewcommand{\bibname}{Bibliograf�a}
\renewcommand{\indexname}{�ndice Alfab�tico}
\renewcommand{\figurename}{Figura}
\renewcommand{\partname}{Parte}
\renewcommand{\chaptername}{Cap�tulo}
\renewcommand{\appendixname}{Ap�ndice}
\renewcommand{\abstractname}{Resumen}
\renewcommand{\tablename}{Tabla}

\maketitle
\newpage
\tableofcontents
\newpage
\listoffigures
\newpage
\listoftables
\newpage

\chapter*{Pre�mbulo}

\htmladdnormallink{GOsa}{http://gosa.gonicus.de}, es un proyecto creado en el a�o 2001 por Cajus Pollmeier \htmladdnormallink{Cajus Pollmeier}{mailto://pollmeier@gonicus.de}

En su versi�n 1.0 era un proyecto ambicioso, pero mal enfocado. Entre como desarrollador del proyecto en Junio del 2003 en las primeras versiones de la versi�n 1.99.xx.
El c�digo fue rehecho completamente y se creo una nueva versi�n modular y extensible (basada en plugins) y se optimizo enormemente su funcionamiento.

La versi�n actual de GOsa (a la fecha de la versi�n de este documento) es la 2.3.
Es capaz de gestionar gran cantidad de servicios como samba2/3, pureftpd, postfix, cyrus-imap, posix, etc.
Y cuenta con un peque�o, pero muy activo grupo de desarrolladores dirigido por Cajus.

\chapter{Introducci�n}
La administraci�n de sistemas puede llegar a ser una tarea realmente complicada, demasiados usuarios con servicios y permisos diferentes.
\section{Copyright and Disclaimer}
Copyright (c) 2005 Alejandro Escanero Blanco. Permission is granted to copy, distribute and/or modify this document under the terms of the GNU Free Documentation License, Version 1.2 or any later version published by the Free Software Foundation; with no Invariant Sections, with no Front-Cover Texts and with no Back-Cover Texts. A copy of the license is included in the section entitled "GNU Free Documentation License".

If you have questions, please visit the following url: http://www.gnu.org/licenses/fdl.txt\\ And contact at: \htmladdnormallink{blainett@yahoo.es}{mailto://blainett@yahoo.es}

\chapter{Seguridad y Certificados}
\section{Introduction SSL}
\section{Creaci�n de certificados}
La seguridad es uno de los puntos mas importantes al configurar un servidor, necesitaremos un entorno seguro donde no permitir que los usuarios manipulen y accedan a codigo o programas.

Una formas de conseguir esto es usando encriptaci�n, con lo que buscamos que los usuarios y el servidor se comuniquen de forma que nadie mas pueda acceder a los datos. Esto se consigue con encriptaci�n.

La otra manera de asegurar el sistema es que si existe alg�n fallo en el sistema o en el c�digo, y un intruso intenta ejecutar codigo, este se vea incapacitado, ya que existen poderosas limitaciones, como no permitir que ejecute comandos, lea el codigo de otros script, no pueda modificar nada y tenga un usuario con muy limitados recursos.
\subsection{ Certificados SSL}

\noindent Existe amplia documentaci�n sobre encriptaci�n y concretamente sobre SSL, un sistema de encriptaci�n con clave publica y privada.\\
\\
\noindent Como el paquete openSSL ya lo tenemos instalado a partir de los pasos anteriores, debemos crear los certificados que usaremos en nuestro servidor web.\\
\\
\noindent Supongamos que guardamos el certificado en /etc/apache2/ssl/gosa.pem\\
\\
\begin{tabular}{|l|}\hline 
\#>FILE=/ect/apache2/ssl/gosa.pem\\
\#>export RANDFILE=/dev/random\\
\#>openssl req -new -x509 -nodes -out \$FILE -keyout /etc/apache2/ssl/apache.pem\\
\#>chmod 600 \$FILE\\
\#>ln -sf \$FILE /etc/apache2/ssl/`/usr/bin/openssl x509 -noout -hash < \$FILE`.0\\
\hline \end{tabular}
\vspace{0.5cm}
\\
\noindent Con esto hemos creado un certificado que nos permite el acceso SSL a nuestras p�ginas.\\
\\
\noindent Si lo que queremos es una configuraci�n que nos permita no solo que el tr�fico est� encriptado, sino que adem�s el cliente garantice que es un usuario v�lido, debemos provocar que el servidor pida una certificaci�n de cliente. \\
\\
\noindent En este caso seguiremos un procedimiento mas largo, primero la creaci�n de una certificaci�n de CA:\\
\\
\begin{tabular}{|l|}\hline 
\#>CAFILE=gosa.ca\\
\#>KEY=gosa.key\\
\#>REQFILE=gosa.req\\
\#>CERTFILE=gosa.cert\\
\#>DAYS=2048\\
\#>OUTDIR=.\\
\#>export RANDFILE=/dev/random\\
\#>openssl req -new -x509 -keyout \$KEY -out \$CAFILE -days \$DAYS\\
\hline \end{tabular}
\vspace{0.5cm}
\\
\noindent Despu�s de varias cuestiones tendremos una CA, ahora hacemos un requerimiento para un nuevo certificado:\\
\\
\begin{tabular}{|l|}\hline 
\#>DAYS=365\\
\#>openssl req -new -keyout \$REQFILE -out \$REQFILE -days\$DAYS\\
\hline \end{tabular}
\vspace{0.5cm}
\\
\noindent Creamos una configuraci�n para usar la CA con openssl y la guardamos en openssl.cnf:\\
\\
\begin{tabular}{|l|}\hline 
\verb|HOME = .|\\
\verb|RANDFILE = $ENV::HOME/.rnd|\\
\verb|[ ca ]|\\
\verb|default_ca  = CA_default|\\
\verb|[ CA_default ]|\\
\verb|dir = .|\\
\verb|database = index.txt|\\
\verb|serial = serial|\\
\verb|default_days = 365|\\
\verb|default_crl_days= 30|\\
\verb|default_md = md5|\\
\verb|preserve = no|\\
\verb|policy = policy_anything|\\
\verb|[ policy_anything ]|\\
\verb|countryName = optional|\\
\verb|stateOrProvinceName  = optional|\\
\verb|localityName = optional|\\
\verb|organizationName = optional|\\
\verb|organizationalUnitName  = optional|\\
\verb|commonName = supplied|\\
\verb|emailAddress = optional|\\
\hline \end{tabular}
\vspace{0.5cm}
\\
\noindent Firmamos el nuevo certificado:\\
\\
\begin{tabular}{|l|}\hline 
\#>\verb|touch index.txt|\\
\#>\verb|touch index.txt.attr|\\
\#>\verb|echo "01" >serial|\\
\#>\verb|openssl ca -config openssl.cnf -policy policy_anything \|\\\verb|-keyfile $KEY -cert $CAFILE -outdir . -out $CERFILE -infiles $REQFILE|\\
\hline \end{tabular}
\vspace{0.5cm}
\\
\noindent Y creamos un pkcs12 para configurar la certificaci�n en los clientes:\\
\\
\begin{tabular}{|l|}\hline 
\#>openssl pkcs12 -export -inkey \$KEY -in \$CERTFILE -out certificado\_cliente.pkcs12\\
\hline \end{tabular}
\vspace{0.5cm}
\\
\noindent Este certificado se puede instalar en el cliente, y en el servidor mediante la configuraci�n explicada en cada uno, esto nos dar� la seguridad de que su comunicaci�n ser� estrictamente confidencial.\\


\chapter{Servicios de seguridad - Kerberos}

\section{Seguridad e identificaci�n}
�Qui�n se conecta al servidor?\\
�Puedo estar seguro de que se puede confiar en el cliente, y el cliente en el servidor?

Esto es solo un peque�o resumen, para mas documentaci�n vease Criptograf�a y Seguridad en Computadores\cite{cripto1} en espa�ol.

\subsection{Caso 1: Las contrase�as van en texto plano}
Est�n ah�, todo aquel que vea el tr�fico de la red las ver�.
Solo es factible si se est�n usando canales que se consideren seguros (SSL,ipsec,etc).

\subsection{El problema del hombre de enmedio}
Si alguien ve tu usuario y tu contrase�a puede hacer algo peor que simplemente ver que haces, puede suplantarte.

El hombre de enmedio (man in the middle) es un sistema que esta entre el cliente y el servidor que coge las peticiones del cliente, las manipula y las envia al servidor, por supuesto tambien puede manipular lo que viene del servidor.

\subsection{Caso 2: Las contrase�as tienen codificaci�n simetrica}

\begin{enumerate}
\item Mejora mucho la seguridad, cuanto mejor sea la encriptaci�n sera mas seguro.
\item A�n as� es problem�tico y deberia usarse bajo canales seguros.
\item �Como enviamos la clave con la que encriptamos la contrase�a?
\item Si esta clave cae, se producira una situaci�n como la del envio en texto plano, y volvemos a estar en situaci�n de que un sistema intermedio tome nuestra personalidad.
\item Se considera segura a partir de 128bits de longitud.
\item No autentifica quien envio el mensaje.
\end{enumerate}

\subsection{Caso 3: Cifrado por bloques (Hashes)}

\begin{enumerate}
\item Las contrase�as se codifican de tal manera que no se puede volver a conseguir la contrase�a por otro metodo que no sea la fuerza bruta.
\item Es menos problem�tico, pero deberia usarse solo bajo canales seguros.
\item Se env�an de esta manera por la red, cualquiera puede identificar que es una contrase�a e intentar romper por fuerza bruta la contrase�a.
\item Sigue siendo sensible al problema del robo de identidad, ya que no autentifica quien envio el mensaje.
\item Se puede mejorar usando tecnicas de desafio, enviando antes de pedir la clave un codigo al usuario con el que mejorar la seguridad del c�digo resultante.
\end{enumerate}

\subsection{Caso 4: Cifrado Asimetrico, Certificados SSL}
\begin{enumerate}
\item Se dividen en dos partes: la privada, aquella con la que codificamos, y la publica, que es aquella con la que decodifican los mensajes los clientes.
\item Necesita logitudes de clave muy largas para ser seguros (>1024bits) y mucho mas tiempo de computaci�n que los cifrados simetricos.
\item En la pr�ctica se utilizan para el envio de la clave y despues los mensaje se envian con codificaci�n simetrica.
\item Es mas resistente a sistemas intermedios, ya que este no puede acceder a la clave privada y por lo tanto no puede codificar mensajes.
\end{enumerate}


\subsection{Caso 5: Kerberos para identificaci�n}
\begin{enumerate}
\item El protocolo supone que la red es insegura y que hay sistemas intermedios que pueden escuchar.
\item Los usuarios y servicios (principales) deben autentificarse ante un tercero, el servidor kerberos, el cual es aceptado como autentico.
\item Usa cifrado simetrico y asimetrico convirtiendo el conjunto en una red segura.
\end{enumerate}


\section{El protocolo Kerberos}

\subsection{El servidor Kerberos}
El servidor kerberos no sirve un unico servicio, sino tres: \\
AS = Servidor de autentificaci�n.\\
TGS = Servidor de Tickets.\\
SS = Servidor de servicios.\\

\subsection{Clientes y Servidores}

El cliente autentifica contra el AS, despues demuestra al TGS que esta autorizado para recibir el ticket para el servicio que quiere usar, y por �ltimo demuestra ante el SS que esta autorizado para usar el servicio.

\subsection{Que es un ticket y como funciona Kerberos}

\begin{enumerate}
\item[1.-] Un usuario introduce una clave y contrase�a en el cliente.
\item[2.-] El Cliente usa un hash sobre la contrase�a y la convierte en la clave secreta del cliente.
\item[3.-] El cliente env�a un mensaje al AS pidiendo servicio para el cliente.
\item[4.-] El AS comprueba si el usuario existe en la base de datos. Si existe le envia dos mensajes al cliente.\\
El mensaje A: Tiene una clave de sesi�n del TGS codificada usando la clave secreta del usuario.\\
El mensaje B: (TGT)Ticket-Granting Ticket. Este incluye el identificador del cliente, la direcci�n de red es este, un periodo de valided, y la clave de sesi�n del TGS, todo codificado usando la clave secreta del TGS.
\item[5.-] El cliente recibe los mensajes A y B, con su clave secreta decodifica el mensaje A y coge la clave de sesi�n del TGS, con esta podra comunicarse con el TGS. Se observa que el cliente no puede decodificar el mensaje B al no tener la clave secreta del TGS.
\item[6.-] Cuando el cliente quiere usar alg�n servicio, envia los siguientes mensajes al TGS:\\
Mensaje C: Compuesto por el TGT del mensaje B y el identificador de petici�n de servicio.\\
Mensaje D: Autentificador (El cual est� compuesto por el identificador del cliente y una marca horaria - timestamp -) codificado con la clave de sesi�n del TGS.
\item[7.-] Al recibir los dos mensajes, el TGS decodifica el autentificador usando la clave de sesi�n del usuario y envia los siguientes mensajes al cliente:\\
Mensaje E: Ticket Cliente Servidor, que contiene el identificador del cliente, su direcci�n de red, un periodo de valided, y la clave de sesi�n del TGS, codificado usando la clave secreta del servidor.\\
Mensaje F: Clave de sesi�n Cliente / Servidor codificada con la clave de sesi�n del TGS del cliente.
\item[8.-] Una vez recibidos desde el TGS los mensajes, el cliente tiene informaci�n suficiente para autentificarse ante el SS. El cliente se conectara al SS y le enviara los siguientes mensajes:\\
Mensaje G: El ticket de cliente / servidor codificado con la clave secreta del servidor. (Mensaje E).\\
Mensaje H: Un nuevo autentificador que contiene el identificador del cliente, una marca horaria y que esta codificado usando la clave de sesi�n cliente / servidor.
\item[9.-] El SS decodifica el Mensaje G usando su propia clave secreta y usando la clave Cliente/TGS en el mensaje F consigue la clave de sesion cliente/servidor, entonces le enviara el siguiente mensaje al cliente para confirmar su identidad:\\
Mensaje I: La marca horaria del Autentificador mas 1, codificado usando la clave de sesi�n cliente/servidor.
\item[10.-] El cliente decodifica el mensaje de confirmaci�n y comprueba si la marca horaria ha sido actualizada correctamente. Si todo es correcto, el cliente confiara en el servidor y puede comenzar a hacer peticiones al servidor.
\item[11.-] El servidor responde a las peticiones de ese cliente que ha sido autentificado.

\end{enumerate}

\section{MIT Kerberos}

\subsection{Instalaci�n}

\subsection{Configuraci�n y funcionamiento}

\subsection{Repliaci�n - kprop}

\subsection{Ventajas y desventajas}

\section{El servidor Heimdal Kerberos}

\subsection{Instalaci�n}

\subsection{Configuraci�n y funcionamiento}

\subsection{Repliaci�n - hprop}

\subsection{Repliaci�n incremental - iprop}

\subsection{Heimdal sobre ldap}

\subsection{Ventajas y desventajas}

\section{La configuraci�n de SASL}

\subsection{Modulos para kerberos}

\section{La configuraci�n de clientes MS Windows}



\include{manual_gosa_es_ldap}

\chapter{Apache y PHP}
\section{Introducci�n a Apache}

GOsa es una aplicaci�n escrita en el lenguaje de programaci�n PHP y pensada para uso a trav�s de p�ginas web.

Aunque todo el mundo conoce lo que es una p�gina web, no viene de menos repasar algunos puntos:

\begin{description}
\item[WWW]
La World Wide Web (Red alrededor del mundo) es el motor de lo que conocemos como internet, es un espacio de informaci�n donde cada recurso est� identificado por su URI (Identificador de Recurso Universal /  Universal Resource Identifier), este define el protocolo necesario para acceder a la informaci�n, el equipo que la posee y donde est� colocada.

La WWW es la gran revoluci�n de nuestra �poca, es una fuente enorme de informaci�n. Y como tal todas las aplicaciones est�n siendo orientadas a ella. GOsa usa WWW por una sencilla raz�n, distribuye el programa, una aplicaci�n orientada a internet es capaz de ser usada desde cualquier lugar y pr�cticamente en cualquier momento. GOsa no necesita estar siendo ejecutado en la misma m�quina que lo tiene, mas 
aun cada uno de los servidores que controla pueden estar en m�quinas diferentes y en remotos lugares.

\item[HTTP]
\htmladdnormallink{HTTP}{http://www.w3.org/Protocols/}\cite{2616} es el acr�nimo de Protocolo de Transferencia de Texto / HyperText Transfer Protocol, cuyo prop�sito mas importante es la publicaci�n y recepci�n de "p�ginas Web".

Es un protocolo de nivel de aplicaci�n ideado para sistemas distribuidos de informaci�n hipermedia. Ha estado siendo usada para la WWW desde 1990, la versi�n actual es HTTP/1.1.

El funcionamiento pr�ctico se puede reducir a un cliente que realiza una petici�n y a un servidor que gestiona esa petici�n y realiza una respuesta.

\item[HTML]
Si la petici�n del cliente y la respuesta del servidor son correctas, la respuesta del servidor contendr� alg�n tipo de hipermedia, el mas habitual es \htmladdnormallink{HTML}{http://www.w3.org/TR/1998/REC-html40-19980424/} (Lenguaje de marcas de hipertexto /  HyperText Markup Language), un lenguaje pensado para la publicaci�n con contenidos y para una f�cil navegaci�n por ellos. Es un protocolo en constante desarrollo, la versi�n actual es HTML4.01 y en publicaci�n XHTML2.0
\end{description}

\htmladdnormallink{APACHE}{http://httpd.apache.org/} es el servidor HTTP mas utilizado que \htmladdnormallink{ existe }{http://news.netcraft.com/archives/web_server_survey.html}, seguro, eficiente y extensible.

En este manual nos centraremos en este servidor, ya que es el mas usado y tiene una licencia calificada de opensource.

Mas informaci�n sobre este servidor en \htmladdnormallink{http://httpd.apache.org/docs-2.0/}{http://httpd.apache.org/docs-2.0/}
 
\section{Introducci�n a PHP}

PHP (PHP: Hypertext Preprocessor), es un lenguaje de interpretado alto nivel, especialmente pensado para el dise�o de p�ginas webs. Su sintaxis es una mezcla de C, Perl y Java. Es embebido en las p�ginas HTML y es ejecutado por el servidor HTTP.

PHP est� ampliamente extendido y tiene un numeroso grupo de desarrolladores, una \htmladdnormallink{ extensa documentaci�n }{http://www.php.net/docs.php} y numerosos sitios webs con documentaci�n y ejemplos.

\newpage

\section{Instalaci�n }
\subsection{Descargando e Instalando Apache}
\label{down_apache}
Al igual que en el cap�tulo anterior, Apache est� en pr�cticamente todas las distribuciones, aunque veremos su instalaci�n desde las fuentes. Nos vamos a centrar por ahora en las versiones mas avanzadas de apache, la serie 2.0.XX considerada estable.

Se recomienda instalar los mismos paquetes que se necesitan para openLDAP\ref{down_ldap}.

Se puede descargar de: \htmladdnormallink{http://httpd.apache.org/download.cgi}{http://httpd.apache.org/download.cgi}, la versi�n que vamos a descargar en /usr/src es la httpd-2.0.XX.tar.gz

Ejecutamos \htmladdnormallink{./configure}{http://warping.sourceforge.net/gosa/contrib/es/configure-apache.sh} con las siguientes opciones.

\begin{itemize}
\item[]Generales\\
\begin{tabular}{|ll|}\hline 
--enable-so & $\rightarrow$ Soporte de Objetos Din�micos Compartidos (DSO)\\
--with-program-name=apache2 & \\
--with-dbm=db42 & $\rightarrow$ Versi�n de la Berkeley DB que vamos a usar\\
--with-external-pcre=/usr & \\
--enable-logio & $\rightarrow$ Registro de entrada y salida\\
--with-ldap=yes & \\
--with-ldap-include=/usr/include & \\
--with-ldap-lib=/usr/lib & \\
\hline \end{tabular}
 
\item[]Soporte suexec\\
\begin{tabular}{|ll|}\hline
--with-suexec-caller=www-data & \\
--with-suexec-bin=/usr/lib/apache2/suexec2 & \\
--with-suexec-docroot=/var/www & \\
--with-suexec-userdir=public\_html & \\
--with-suexec-logfile=/var/log/apache2/suexec.log & \\
\hline \end{tabular}

\item[]
\begin{longtable}{|ll|}
\hline
\multicolumn{2}{|c|}{\textbf{M�dulos}}\\
\hline
\endfirsthead
\hline
\endhead
\hline
\multicolumn{2}{|c|}{Continue $\ldots$}\\
\hline
\endfoot
\hline
\multicolumn{2}{|c|}{\textbf{End}}\\
\hline
\endlastfoot
--enable-userdir=shared & $\rightarrow$ mod\_userdir, m�dulo para directorios de usuario\\
--enable-ssl=shared & $\rightarrow$ mod\_ssl, m�dulo de conectividad segura SSL\\
--enable-deflate=shared & $\rightarrow$ mod\_deflate, m�dulo para comprimir la informaci�n enviada\\
--enable-ldap=shared & $\rightarrow$ mod\_ldap\_userdir, m�dulo para cach� y conexiones ldap\\
--enable-auth-ldap=shared & $\rightarrow$ mod\_ldap, m�dulo de autentificaci�n en ldap\\
--enable-speling=shared & $\rightarrow$ mod\_speling, m�dulo para la correcci�n de fallos en URL\\
--enable-include=shared & $\rightarrow$ mod\_include, m�dulo para la inclusi�n de otras configuraciones\\
--enable-rewrite=shared & $\rightarrow$ mod\_rewrite, permite la manipulaci�n de URL\\
--enable-cgid=shared & $\rightarrow$ CGI script\\
--enable-vhost-alias=shared & $\rightarrow$ m�dulo de alias de dominios virtuales\\
--enable-info=shared & $\rightarrow$ Informaci�n del servidor\\
--enable-suexec=shared & $\rightarrow$ Cambia el usuario y el grupo de los procesos\\
--enable-unique-id=shared & $\rightarrow$ Identificador �nico por petici�n\\
--enable-usertrack=shared & $\rightarrow$ Seguimiento de la sesi�n de usuario\\
--enable-expires=shared & $\rightarrow$ M�dulo para el env�o de la cabecera de expiraci�n\\
--enable-cern-meta=shared & $\rightarrow$ Ficheros meta tipo CERN\\
--enable-mime-magic=shared & $\rightarrow$ Determina autom�ticamente el tipo MIME\\
--enable-headers=shared & $\rightarrow$ Control cabeceras HTTP\\
--enable-auth-anon=shared & $\rightarrow$ Acceso a usuarios an�nimos\\
--enable-proxy=shared & $\rightarrow$ Permite el uso de Apache como proxy\\
--enable-dav=shared & $\rightarrow$ Capaz de manejar el protocolo WebDav\\
--enable-dav-fs=shared & $\rightarrow$ Proveedor DAV para el sistema de archivos\\
--enable-auth-dbm=shared & $\rightarrow$ Autentificaci�n basada en base de datos DBM\\
--enable-cgi=shared & $\rightarrow$ Permite CGI scripts\\
--enable-asis=shared & $\rightarrow$ Tipos de archivos como son\\
--enable-imap=shared & $\rightarrow$ Mapas de im�genes en el lado de servidor\\
--enable-ext-filter=shared & $\rightarrow$ M�dulo para filtros externos\\
--enable-authn-dbm=shared & \\
--enable-authn-anon=shared & \\
--enable-authz-dbm=shared & \\
--enable-auth-digest=shared & $\rightarrow$ Colecci�n de autentificaciones seg�n RFC2617\\
--enable-actions=shared & $\rightarrow$ Activa acciones seg�n peticiones\\
--enable-file-cache=shared & $\rightarrow$ Cache de archivos\\
--enable-cache=shared & $\rightarrow$ Cache din�mico de archivos\\
--enable-disk-cache=shared & $\rightarrow$ Cache de disco\\
--enable-mem-cache=shared & $\rightarrow$ Cache de memoria\\
\hline \end{longtable}
\end{itemize}

Una vez configurado, hacemos:\\
\\
\begin{tabular}{|l|}\hline 
\#make \&\& make install\\
\hline \end{tabular}
\newpage


\subsection{ Instalando PHP sobre Apache}

Se puede descargar de \htmladdnormallink{http://www.php.net/downloads.php}{http://www.php.net/downloads.php} siendo la versi�n necesaria a la fecha de este manual para utilizar GOsa la 4.3.XX, ya que las versiones 5.0.XX a�n no est�n soportadas. Las descargaremos en /usr/src.

Para poder compilar los m�dulos necesarios adem�s de necesitar las librer�as de desarrollo de la seccion Servidores \ref{servidores}, adem�s de las mismas que para openLDAP\ref{down_ldap} y Apache\ref{down_apache} necesitaremos alguna librer�a mas:

\begin{itemize}
\item[libbz2]
La podemos descargar de \htmladdnormallink{http://sources.redhat.com/bzip2/}{http://sources.redhat.com/bzip2/} para m�dulo de compresi�n BZ2.
\item[e2fsprogs]
Se puede descargar de \htmladdnormallink{http://e2fsprogs.sourceforge.net}{http://e2fsprogs.sourceforge.net} para acceso al sistema de archivos.
\item[expat]
Se descarga de \htmladdnormallink{http://expat.sourceforge.net/}{http://expat.sourceforge.net/}, es un parser XML.
\item[zziplib]
Bajarla de \htmladdnormallink{http://zziplib.sourceforge.net/}{http://zziplib.sourceforge.net/}, acceso a archivos ZIP.
\item[zlib]
Desde \htmladdnormallink{http://www.gzip.org/zlib/}{http://www.gzip.org/zlib/} para compresi�n GZ.
\item[file]
Desde \htmladdnormallink{http://www.darwinsys.com/freeware/file.html}{http://www.darwinsys.com/freeware/file.html} control de archivos.
\item[sed]
De \htmladdnormallink{http://www.gnu.org/software/sed/sed.html}{http://www.gnu.org/software/sed/sed.html}, una de las herramientas mas potentes para manipulaci�n de texto.
\item[libcurl]
Potente herramienta para manejar archivos remotos, la bajaremos de \htmladdnormallink{http://curl.haxx.se/}{http://curl.haxx.se/} .
\item[gettext]
Herramienta GNU para soporte de varios idiomas, la descargamos de \htmladdnormallink{http://www.gnu.org/software/gettext/gettext.html}{http://www.gnu.org/software/gettext/gettext.html} .
\item[libgd]
Para la manipulaci�n y creaci�n de im�genes desde: \htmladdnormallink{http://www.boutell.com/gd/}{http://www.boutell.com/gd/} .
\item[libjpeg]
Manipulaci�n de im�genes JPEG de \htmladdnormallink{http://www.ijg.org/}{http://www.ijg.org/} .
\item[libpng]
Manipulaci�n im�genes PNG de \htmladdnormallink{http://www.libpng.org/pub/png/libpng.html}{http://www.libpng.org/pub/png/libpng.html} .
\item[mcal]
Librer�a para el acceso a Calendarios remotos, se baja de \htmladdnormallink{http://mcal.chek.com/}{http://mcal.chek.com/} .
\item[libmysql]
Soporte para la famos�sima base de datos, es imprescindible para php, se baja de \htmladdnormallink{http://www.mysql.com/}{http://www.mysql.com/}
\end{itemize}

Una configuraci�n recomendada ser�:


\begin{itemize}
\item[]Apache2\\
\begin{tabular}{|ll|}\hline 
--prefix=/usr --with-apxs2=/usr/bin/apxs2 & \\
--with-config-file-path=/etc/php4/apache2 & \\
\hline \end{tabular}


 
\item[]Opciones de compilaci�n\\
\begin{tabular}{|ll|}\hline
--enable-memory-limit & \# Compilado con l�mite de memoria\\
--disable-debug & \# Compilar sin s�mbolos de depuraci�n\\
--disable-static & \# Sin librer�as est�ticas\\
--with-pic & \# Usar objetos PIC y no PIC\\
--with-layout=GNU & \\
--enable-sysvsem & \# Soporte sysvmsg \\
--enable-sysvshm & \# Soporte sem�foros System V \\
--enable-sysvmsg & \# Soporte memoria compartida System V \\
--disable-rpath & \# Desactiva poder pasar rutas a librer�as adiciones al binario\\
--without-mm & \# Desactivar el soporte de sesiones por memoria\\
\hline \end{tabular}

\item[]De sesi�n\\
\begin{tabular}{|ll|}\hline
--enable-track-vars & \\
--enable-trans-sid & \\
\hline \end{tabular}

\item[]Soporte\\
\begin{tabular}{|ll|}\hline
--enable-sockets & \# Soporte de sockets\\
--with-mime-magic=/usr/share/misc/file/magic.mime & \\
--with-exec-dir=/usr/lib/php4/libexec & \\
\hline \end{tabular}

\item[]pear\\
\begin{tabular}{|ll|}\hline
--with-pear=/usr/share/php & Donde vamos a instalar PEAR\\
\hline \end{tabular}
  
\item[]Funciones\\
\begin{tabular}{|ll|}\hline
--enable-ctype & Soporte funciones de control de caracteres \\
--with-iconv & Soporte funciones iconv\\
--with-bz2 & Soporte Compresi�n BZ2\\
--with-regex=php & Tipo de librer�a de expresiones regulares\\
--enable-calendar & Funciones para conversi�n de calendario\\
--enable-bcmath & Soporte de matem�ticas de precisi�n arbitraria\\
--with-db4 & DBA: Soporte Berkeley DB versi�n 4\\
--enable-exif & Soporte funciones exif, para lectura metadata JPG y TIFF\\
--enable-ftp & Soporte funciones FTP \\
--with-gettext & Soporte Localizaci�n\\
--enable-mbstring & \\
--with-pcre-regex=/usr & \\
--enable-shmop & Funciones de memoria compartida\\
--disable-xml --with-expat-dir=/usr & Usa el xml de expat en vez del que viene con php\\
--with-xmlrpc & \\
--with-zlib & \\
--with-zlib-dir=/usr & \\
--with-imap=shared,/usr & Soporte imap gen�rico\\
--with-kerberos=/usr & Imap con autentificaci�n kerberos\\
--with-imap-ssl & Imap con acceso seguro SSL\\
--with-openssl=/usr & \\
--with-zip=/usr & \\
--enable-dbx & Capa de abstracci�n a base de datos\\
\hline \end{tabular}

\item[]M�dulos externos\\
\begin{tabular}{|ll|}\hline
--with-curl=shared,/usr & Manejo remoto de archivos\\
--with-dom=shared,/usr --with-dom-xslt=shared,/usr --with-dom-exslt=shared,/usr & Con xmlrpc ya integrado\\
--with-gd=shared,/usr --enable-gd-native-ttf & Soporte de manejo de gr�ficos\\
--with-jpeg-dir=shared,/usr & Soporte GD para jpeg\\
--with-png-dir=shared,/usr & Soporte GD para png\\
--with-ldap=shared,/usr & Soporte para ldap\\
--with-mcal=shared,/usr & Soporte de calendarios\\
--with-mhash=shared,/usr & M�dulo para varios algoritmos de generaci�n de claves\\
--with-mysql=shared,/usr & Soporte de base de datos Mysql\\    
\hline \end{tabular}
\end{itemize}
Posteriormente hacemos:\\
\\
\begin{tabular}{|l|}\hline 
\#make \&\& make install\\
\hline \end{tabular}
\newpage
\section{ Configuraci�n Apache2}

La configuraci�n de apache se guardara en el directorio /etc/apache2 en los siguientes ficheros y directorios:
\begin{itemize}
\item[]Archivo apache2.conf:\\
COnfiguraci�n principal de apache2, tiene la configuraci�n necesaria para arrancar apache.\\
No necesitamos editar este archivo.
\item[]Archivo ports.conf\\
Que puertos escucha apache, necesitamos dos, el puerto 80 para HTTP y el 443 para HTTPS, editaremos el ficheros, dejandolo como esto:
\begin{tabular}{|l|}\hline
Listen 80,443\\
\hline \end{tabular}
\item[]Directorio conf.d:\\
Directorio para configuraciones especiales, no lo necesitamos.
\item[]Directorios mods-available y mods-enabled:\\
Este directorio tiene todos los m�dulos que podemos usar de apache2, para poder usar un m�dulo es necesario enlazar este al directorio mods-enabled.
\item[]Directorios sites-available y sites-enabled:\\
En sites-available debemos configurar los sitios que vamos a usar.\\
Por ejemplo vamos a crear el sitio no seguro gosa, que vamos a usar para redirigir las peticiones a un servidor seguro.

La configuraci�n de GOsa (sites-available/gosa) puede ser parecida a esta::\\
\begin{tabular}{|l|}\hline
\noindent NameVirtual *\\
<VirtualHost *>\\
\verb|    |ServerName gosa.chaosdimension.org\\
\\
\verb|    |Redirect /gosa https://gosa.chaosdimension.org/gosa\\
\\
\verb|    |CustomLog /var/log/apache/gosa.log combined\\
\verb|    |ErrorLog /var/log/apache/gosa.log\\
\\
</VirtualHost>\\
\hline \end{tabular}

Una vez sea guardada, podemos activarla haciendo esto:\\
\\
\begin{tabular}{|l|}\hline 
\#>ln -s /etc/apache2/mods-available/suphp.conf /etc/apache2/mods-enabled/suphp.conf\\
\hline \end{tabular}
\\
\item[]Directorio ssl:\\
Directorio de configuraci�n de Secure Socket Layer, esto lo veremos en la siguiente secci�n.
\end{itemize}
\newpage
\subsection{ Seguridad}

La seguridad es uno de los puntos mas importantes al configurar un servidor apache, necesitaremos un entorno seguro donde no permitir que los usuarios manipulen y accedan a codigo o programas.

La formas de conseguir esto es usando encriptaci�n, con lo que buscamos que los usuarios y el servidor se comuniquen de forma que nadie mas pueda acceder a los datos. Esto se consigue con encriptaci�n.

La otra manera de asegurar el sistema es que si existe alg�n fallo en el sistema o en el c�digo, y un intruso intenta ejecutar codigo, este se vea incapacitado, ya que existen poderosas limitaciones, como no permitir que ejecute comandos, lea el codigo de otros script, no pueda modificar nada y tenga un usuario con muy limitados recursos.
\subsubsection{ Certificados SSL}

\noindent Existe amplia documentaci�n sobre encriptaci�n y concretamente sobre SSL, un sistema de encriptaci�n con clave publica y privada.\\
\\
\noindent Como el paquete openSSL ya lo tenemos instalado a partir de los pasos anteriores, debemos crear los certificados que usaremos en nuestro servidor web.\\
\\
\noindent Supongamos que guardamos el certificado en /etc/apache2/ssl/gosa.pem\\
\\
\begin{tabular}{|l|}\hline 
\#>FILE=/ect/apache2/ssl/gosa.pem\\
\#>export RANDFILE=/dev/random\\
\#>openssl req -new -x509 -nodes -out \$FILE -keyout /etc/apache2/ssl/apache.pem\\
\#>chmod 600 \$FILE\\
\#>ln -sf \$FILE /etc/apache2/ssl/`/usr/bin/openssl x509 -noout -hash < \$FILE`.0\\
\hline \end{tabular}
\\
\noindent Con esto hemos creado un certificado que nos permite el acceso SSL a nuestras p�ginas.\\
\\
\noindent Si lo que queremos es una configuraci�n que nos permita no solo que el tr�fico est� encriptado, sino que adem�s el cliente garantice que es un usuario v�lido, debemos provocar que el servidor pida una certificaci�n de cliente. \\
\\
\noindent En este caso seguiremos un procedimiento mas largo, primero la creaci�n de una certificaci�n de CA:\\
\\
\begin{tabular}{|l|}\hline 
\#>CAFILE=/ect/apache2/ssl/gosa.ca\\
\#>KEY=/etc/apache2/ssl/gosa.key\\
\#>REQFILE=/etc/apache2/ssl/gosa.req\\
\#>CERTFILE=/ect/apache2/ssl/gosa.cert\\
\#>DAYS=365\\
\#>export RANDFILE=/dev/random\\
\#>openssl req -x509 -keyout \$CAKEY -out \$CAFILE \$DAYS\\
\hline \end{tabular}
\\
\noindent Despu�s de varias cuestiones tendremos una CA, ahora hacemos un requerimiento a la CA creada:\\
\\
\begin{tabular}{|l|}\hline 
\#>openssl req -new -keyout \$REQFILE -out \$REQFILE \$DAYS\\
\hline \end{tabular}
\\
\noindent Firmamos el nuevo certificado:\\
\\
\begin{tabular}{|l|}\hline 
\#>openssl ca -policy policy\_anything -out \$CERFILE -infiles \$REQFILE\\
\hline \end{tabular}
\\
\noindent Y creamos un pkcs12 para configurar la certificaci�n en los clientes:\\
\\
\begin{tabular}{|l|}\hline 
\#>openssl pkcs12 -export -inkey \$KEY -in \$CERTFILE -out certificado\_cliente.pkcs12\\
\hline \end{tabular}
\\
\noindent Este certificado se puede instalar en el cliente, y en el servidor web mediante la configuraci�n explicada en el siguiente punto, nos dar� la seguridad de que solo acceder�n aquellos clientes que nosotros deseamos y que su comunicaci�n ser� estrictamente confidencial.\\



\subsubsection{ Configurando mod-SSL}


\noindent El m�dulo SSL viene de serie con apache2, esto simplificara nuestro trabajo. Para saber si est� ya configurado:\\
\\
\begin{tabular}{|l|}\hline 
\#> if [ -h /etc/apache2/mods-enabled/ssl.load ]; then echo "m�dulo instalado";else echo "m�dulo no instalado"; fi\\
\hline \end{tabular}
\\
\noindent Para activarlo haremos lo siguiente:\\
\\
\begin{tabular}{|l|}\hline 
\#>ln -s /etc/apache2/mods-available/ssl.conf /etc/apache2/mods-enabled/ssl.conf\\
\#>ln -s /etc/apache2/mods-available/ssl.load /etc/apache2/mods-enabled/ssl.load\\
\hline \end{tabular}
\\
\noindent Esto configurar� el m�dulo en apache2 y se podr� utilizar despu�s de recargar el servidor con:\\
\\
\begin{tabular}{|l|}\hline 
\#>/etc/init.d/apache2 restart\\
\hline \end{tabular}
\\
\\
\noindent Para el caso de querer solo una configuraci�n para comunicaci�n encriptada, crearemos en /etc/apache2/sites-available, gosa-ssl:\\
\\
\begin{tabular}{|l|}\hline 
\noindent NameVirtual *:443\\
<VirtualHost *:443>\\
\verb|    |ServerName gosa.chaosdimension.org\\
\\
\verb|    |alias /gosa /usr/share/gosa/html\\
\\
\verb|    |DocumentRoot /var/www/gosa.chaosdimension.org\\
\verb|    |CustomLog /var/log/apache/gosa.log combined\\
\verb|    |ErrorLog /var/log/apache/gosa.log\\
\\
\verb|    |SSLEngine On\\
\verb|    |SSLCertificateFile    /etc/apache2/ssl/gosa.cert\\
\verb|    |SSLCertificateKeyFile /etc/apache2/ssl/gosa.key\\
\verb|    |SSLCertificateChainFile /etc/apache2/ssl/gosa.cert\\
\verb|    |SSLCertificateKeyFile /etc/apache2/ssl/gosa.key\\
\verb|    |SSLCACertificateFile /etc/apache2/ssl/gosa.ca\\
\verb|    |SSLCACertificatePath /etc/apache2/ssl/\\
\verb|    |SSLLogLevel error\\
\verb|    |SSLLog /var/log/apache2/ssl-gosa.log\\
\\
</VirtualHost>\\
\hline \end{tabular}
\\
\noindent Para una comunicaci�n encriptada en la cual verificamos el certificado del cliente:
\\
\begin{tabular}{|l|}\hline 
\noindent NameVirtual *:443\\
<VirtualHost *:443>\\
\verb|    |ServerName gosa.chaosdimension.org\\
\\
\verb|    |alias /gosa /usr/share/gosa/html\\
\\
\verb|    |DocumentRoot /var/www/gosa.chaosdimension.org\\
\verb|    |CustomLog /var/log/apache/gosa.log combined\\
\verb|    |ErrorLog /var/log/apache/gosa.log\\
\\
\verb|    |SSLEngine On\\
\verb|    |SSLCertificateFile    /etc/apache2/ssl/gosa.cert\\
\verb|    |SSLCertificateKeyFile /etc/apache2/ssl/gosa.key\\
\verb|    |SSLCertificateChainFile /etc/apache2/ssl/gosa.cert\\
\verb|    |SSLCertificateKeyFile /etc/apache2/ssl/gosa.key\\
\verb|    |SSLCACertificateFile /etc/apache2/ssl/gosa.ca\\
\verb|    |SSLCACertificatePath /etc/apache2/ssl/\\
\verb|    |SSLLogLevel error\\
\verb|    |SSLLog /var/log/apache2/ssl-gosa.log\\
\\      
\verb|    |<Directory /usr/share/gosa >\\
\verb|    |\verb|    |SSLVerifyClient require\\
\verb|    |\verb|    |SSLVerifyDepth 1\\
\verb|    |</Directory>\\
</VirtualHost>\\
\hline \end{tabular}

\subsubsection{ Configurando suphp}
\noindent
Suphp es un m�dulo para apache y php que permite ejecutar procesos de php con un usuario diferente del que usa apache para ejecutar las p�ginas html y php.

Consta de dos partes, una es un modulo para apache que "captura" las peticiones de p�ginas php, comprueba el usuario del archivo, su grupo, y env�a la informaci�n a la otra parte, que es un ejecutable suid-root que lanza php4-cgi con el usuario que le ha sido indicado, este devuelve el resultado al m�dulo del apache.

La idea es minimizar el da�o que se provocar�a al ser explotado un posible fallo del sistema, de esta manera el usuario entrar�a en el sistema con una cuenta no habilitada, sin permisos de ejecuci�n y sin posibilidad de acceso a otro c�digo o sitios web.

Suphp se puede descargar de \htmladdnormallink{http://www.suphp.org/Home.html}{http://www.suphp.org/Home.html}, descomprimiendo el paquete en /usr/src y compilando con las siguientes opciones:
\\
\begin{tabular}{|l|}\hline 
\#>./configure --prefix=/usr \textbackslash \\
\verb|    |--with-apxs=/usr/bin/apxs2 \textbackslash \\
\verb|    |--with-apache-user=www-data \textbackslash \\
\verb|    |--with-php=/usr/lib/cgi-bin/php4 \textbackslash \\
\verb|    |--sbindir=/usr/lib/suphp \textbackslash \\
\verb|    |--with-logfile=/var/log/suphp/suphp.log \textbackslash \\
\verb|    |-with-setid-mode \textbackslash \\
\verb|    |--disable-checkpath \\
\hline \end{tabular}
\\
\noindent Por supuesto necesitaremos tener compilado php para cgi, esto significa volver a compilar php, pero quitando la configuraci�n para apache2 y a�adiendo:\\
\\
\begin{tabular}{|l|}\hline 
\verb|    |--prefix=/usr --enable-force-cgi-redirect --enable-fastcgi \textbackslash\\
\verb|    |--with-config-file-path=/etc/php4/cgi\\
\hline \end{tabular}

\noindent Para configurarlo en apache haremos igual que para ssl, primero comprobamos si est� configurado:\\
\\
\begin{tabular}{|l|}\hline 
\#> if [ -h /etc/apache2/mods-enabled/suphp.load ]; then echo "m�dulo instalado";else echo "m�dulo no instalado"; fi\\
\hline \end{tabular}
\\
\noindent Para activarlo haremos lo siguiente:\\
\\
\begin{tabular}{|l|}\hline 
\#>ln -s /etc/apache2/mods-available/suphp.conf /etc/apache2/mods-enabled/suphp.conf\\
\#>ln -s /etc/apache2/mods-available/suphp.load /etc/apache2/mods-enabled/suphp.load\\
\hline \end{tabular}
\\
\noindent Esto configurar� el m�dulo en apache2 y se podr� utilizar despu�s de recargar el servidor con:\\
\\
\begin{tabular}{|l|}\hline 
\#>/etc/init.d/apache2 restart\\
\hline \end{tabular}
\\

\noindent La configuraci�n del sitio seguro con suphp incluido quedar�a as�:\\
\\
\begin{tabular}{|l|}\hline 
\noindent NameVirtual *:443\\
<VirtualHost *:443>\\
\verb|    |ServerName gosa.chaosdimension.org\\
\\
\verb|    |DocumentRoot /usr/share/gosa/html\\
\verb|    |alias /gosa /usr/share/gosa/html\\
\verb|    |CustomLog /var/log/apache/gosa.log combined\\
\verb|    |ErrorLog /var/log/apache/gosa.log\\
\\
\verb|    |suPHP\_Engine on\\
\\
\verb|    |SSLEngine On\\
\verb|    |SSLCertificateFile    /etc/apache2/ssl/gosa.cert\\
\verb|    |SSLCertificateKeyFile /etc/apache2/ssl/gosa.key\\
\verb|    |SSLCertificateChainFile /etc/apache2/ssl/gosa.cert\\
\verb|    |SSLCertificateKeyFile /etc/apache2/ssl/gosa.key\\
\verb|    |SSLCACertificateFile /etc/apache2/ssl/gosa.ca\\
\verb|    |SSLCACertificatePath /etc/apache2/ssl/\\
\verb|    |SSLLogLevel error\\
\verb|    |SSLLog /var/log/apache2/ssl-gosa.log\\
\\      
\verb|    |<Directory /usr/share/gosa >\\
\verb|    |\verb|    |SSLVerifyClient require\\
\verb|    |\verb|    |SSLVerifyDepth 1\\
\verb|    |</Directory>\\
</VirtualHost>\\
\hline \end{tabular}

\noindent Debemos decidir que usuario vamos a usar, en este caso voy a crear uno llamado gosa, que me sirva para el fin indicado anteriormente:\\
\\
\begin{tabular}{|l|}\hline 
\verb|    |\#useradd -d /usr/share/gosa/html gosa\\
\verb|    |\#passwd -l gosa\\
\verb|    |\#cd /usr/share/gosa\\
\verb|    |\#find /usr/share/gosa -name "*.php" -exec chown gosa {} ";"\\
\verb|    |\#find /usr/share/gosa -name "*.php" -exec chmod 600 {} ";"\\
\hline \end{tabular}



\section{Configuraci�n Php4}

La configuraci�n para mod\_php se guardar� en el sitio que hallamos puesto en la partes anteriores. En nuestro caso es /etc/php4/apache2.

El archivo de configuraci�n siempre es php.ini y en el configuramos los m�dulos.

Una configuraci�n b�sica ser� como esta:
\begin{center}
\begin{longtable}{|l|}
\caption{PHP4 Configuration}\\
\hline
\multicolumn{1}{|c|}{\textbf{PHP4 Configuration}}\\
\hline
\endfirsthead
\hline
\endhead
\hline
\multicolumn{1}{|c|}{Continue $\ldots$}\\
\hline
\endfoot
\hline
\multicolumn{1}{|c|}{\textbf{End}}\\
\hline
\endlastfoot
; Engine\\
\verb|    |engine   = On ; Activa PHP\\
\verb|    |short\_open\_tag = On ; Permite usar <? para simplificar <?php\\
\verb|    |asp\_tags  = Off ; No permitimos etiquetas estilo ASP: <\% \%>\\
\verb|    |precision  = 14 ; N�mero de d�gitos significantes mostrados en n�meros en coma flotante\\
\verb|    |output\_buffering = Off ; Solo permitimos que envie cabecera antes de enviar el contenido.\\
\verb|    |implicit\_flush  = Off ; No forzamos a php a que limpie el buffer de salida despu�s de cada bloque.\\
\\
; Safe Mode\\
\verb|    |\label{sm} safe\_mode  = Off ; No queremos el modo seguro\\
\verb|    |\label{smed} safe\_mode\_exec\_dir = ; Directorio donde se ejecutara PHP\\
\verb|    |\label{smid} safe\_mode\_include\_dir = Directorios donde har� la busqueda PHP de librer�as\\
\verb|    |\label{smaev} safe\_mode\_allowed\_env\_vars = PHP\_     ; Solo se permite a los usuarios\\
\verb|    |\verb|    |\verb|    |;a crear variables del sistema que empiecen por PHP\_\\
\verb|    |\label{smpev} safe\_mode\_protected\_env\_vars = LD\_LIBRARY\_PATH  ; Lista de variables del sistema que\\
\verb|    |\verb|    |\verb|    |; no pueden ser cambiadas por razones de seguridad\\
\verb|    |\label{df} disable\_functions =        ; Funciones que ser�n desactivadas por razones de seguridad\\
\verb|    |\label{auf} allow\_url\_fopen = Yes ; Permitimos que se abran archivos desde PHP\\
\verb|    |\label{ob} open\_basedir = ;\\
\\
; Colores para el modo de s�ntasis coloreada.\\
\verb|    |highlight.string = \#DD0000\\
\verb|    |highlight.comment = \#FF8000\\
\verb|    |highlight.keyword = \#007700\\
\verb|    |highlight.bg  = \#FFFFFF\\
\verb|    |highlight.default = \#0000BB\\
\verb|    |highlight.html  = \#000000\\
\\
; Misc\\
\verb|    |\label{ep}expose\_php = On  ; Indica en el mensaje del servidor web si est� instalado o no.\\
\\
; Resource Limits ;\\
\verb|    |max\_execution\_time = 30     ; Tiempo m�ximo de ejecuci�n del script.\\
\verb|    |memory\_limit = 16M   ; La cantidad m�xima permitida de memoria que puede consumir un script.\\
\\
; Error handling and logging ;\\
\verb|    |error\_reporting = E\_ALL; Indicamos que muestre todos los errores y avisos.\\
\verb|    |display\_errors = Off ; Que no los imprima en pantalla.\\
\verb|    |display\_startup\_errors = Off  ; Que no muestre los errores de arranque de PHP.\\
\verb|    |log\_errors  = On ; Que env�e los errores a un fichero.\\
\verb|    |track\_errors = On ; Que guarde el �ltimo error / aviso para \$php\_errormsg (boolean)\\
\verb|    |error\_log = /var/log/php/php4.log ; Fichero que guardar� los errores\\
\verb|    |warn\_plus\_overloading = Off  ; No avisamos si se usa el operador + con cadenas de texto\\
\\
; Data Handling ;\\
\verb|    |variables\_order  = "EGPCS" ; Esta directiva describe el orden en el cual\\
\verb|    |;se registrar�n las variables de PHP (Siendo G=GET, P=POST, C=Cookie,\\
\verb|    |; E= Sistema, S= Propias de PHP, todas es indicado como EGPCS)\\
\verb|    |\label{rg} register\_globals = Off  ; No queremos que se registren las EGPCS como globales.\\
\verb|    |register\_argc\_argv = Off  ; No declaramos ARGV y ARGC para su uso en scripts.\\
\verb|    |post\_max\_size  = 8M  ; Tama�o m�ximo de un env�o POST que aceptar� PHP.\\

; Magic quotes\\
\verb|    |\label{mqq}magic\_quotes\_gpc = On  ; Comillas a�adidas para gpc(informaci�n GET/POST/Cookie)\\
\verb|    |magic\_quotes\_runtime= Off  ; Comillas a�adidas para informaci�n generada por el sistema, \\
\verb|    |;por ejemplo desde SQL, exec(), etc.\\
\verb|    |magic\_quotes\_sybase = Off  ; Usar comillas a�adidas al estilo de Sybase \\
\verb|    |;(escapa ' con '' en lugar de \textbackslash ')\\
\\
; Tipo de archivo por defecto de PHP y codificaci�n por defecto.\\
\verb|    |default\_mimetype = "text/html"\\
\verb|    |default\_charset = "iso-8859-1"\\
\\
; Rutas y directorios ;\\
\verb|    |\label{ip} include\_path = . ;\\
\verb|    |doc\_root  =     ; Ra�z de las p�ginas php, mejor dejarlo en blanco.\\
\verb|    |user\_dir  =     ; Donde php ejecuta el script, tambien mejor en blanco.\\
\verb|    |;extension\_dir = /usr/lib/php4/apache   ; �Donde estan los m�dulos?\\
\verb|    |enable\_dl  = Off    ; Permitir o no la carga din�mica de m�dulos con la funci�n dl().\\
\\
; Subir ficheros al servidor;\\
\verb|    |file\_uploads = On    ; Permitir el subir archivos al servidor.\\
\verb|    |upload\_max\_filesize = 2M      ; Tama�o m�ximo de los archivos que vamos a subir.\\
\\
; Extensiones din�micas ;\\
\verb|    |extension=gd.so		; Graficos\\
\verb|    |extension=mysql.so	; Mysql\\
\verb|    |extension=ldap.so	; Ldap\\
\verb|    |extension=mhash.so	; Mhash\\
\verb|    |extension=imap.so	; Imap\\
\verb|    |extension=kadm5.so	; Kerberos\\
\verb|    |extension=cups.so	; Cupsys\\
\\
; Log del sistema\\
\verb|[Syslog]|\\
\verb|    |define\_syslog\_variables = Off ; Desactivamos la definici�n de variables de syslog.\\
\\
; funciones de correo\\
\verb|[mail function]|\\
\verb|    |;sendmail\_path =      ;En sistemas Unix, donde esta hubicado sendmail (por defecto es 'sendmail -t -i')\\
\\
; depuraci�n\\
\verb|[Debugger]|\\
\verb|    |debugger.host = localhost ; Donde est� el depurador.\\
\verb|    |debugger.port = 7869	; En que puerto escucha.\\
\verb|    |debugger.enabled = False ; En principio suponemos que no hay depurados.\\
\\
; Opciones SQL\\
\verb|[SQL]|\\
\verb|    |sql.safe\_mode = Off	; Modo seguro de sql, en principio estar� desactivado.\\
\\
; Opciones Mysql\\
\verb|[MySQL]|\\
\verb|    |mysql.allow\_persistent = Off ; Desactivaremos los enlaces persistentes por razones de seguridad.\\
\verb|    |mysql.max\_persistent = -1 ; Numero de conexiones persistentes, no se usa por haberlas desactivado.\\
\verb|    |mysql.max\_links   = -1 ; Numero m�ximo de conexiones, -1 es sin limite.\\
\verb|    |mysql.default\_port  =  3306; Puerto por defecto del mysql.\\
\verb|    |mysql.default\_socket =  ; Nombre de socket que se usaran para conexiones locales MySQL.\\
\verb|    |;Si est� vacio se usara el que tengamos en la configuraci�n de la compilaci�n del PHP.\\
\verb|    |mysql.default\_host  =  ; No configuramos host por defecto.\\
\verb|    |mysql.default\_user  =  ; No configuramos usuario por defecto.\\
\verb|    |mysql.default\_password =  ; No configuramos una contrase�a por defecto.\\
\\
; Control de sesiones\\
\verb|[Session]|\\
\verb|    |session.save\_handler      = files   ; Guardamos la informaci�n de sesi�n en ficheros.\\
\verb|    |\label{ss} session.save\_path         = /var/lib/php4    ; Donde se van ha guardar los ficheros de sesi�n.\\
\verb|    |session.use\_cookies       = 1       ; Usaremos cookies para el seguimiento de sesi�n.\\
\verb|    |session.name              = PHPSESSID   ; Nombre de la sesi�n que ser� usado en el nombre de la cookie.\\
\verb|    |session.auto\_start        = 0       ; No iniciamos sesi�n autom�ticamente.\\
\verb|    |session.cookie\_lifetime   = 0       ; Tiempo de vida de una cookie de sesi�n o 0 si esperamos a que cierre el navegador.\\
\verb|    |session.cookie\_path       = /       ; La ruta para que es v�lida la cookie.\\
\verb|    |session.cookie\_domain     =         ; El dominio para el cual es v�lida la cookie.\\
\verb|    |session.serialize\_handler = php     ; Manipulador usado para serializar los datos.\\
\verb|    |session.gc\_probability    = 1       ; Probabilidad en porcentaje de que el recolector de basura se active en cada sesi�n.\\
\verb|    |session.gc\_maxlifetime    = 1440    ; Despu�s de este tiempo en segundos, la informaci�n guardada\\
\verb|    |; ser� vista como basura para el recolector de basura.\\
\verb|    |session.referer\_check     =         ; Comprueba los Referer HTTP para invalidar URLs externas conteniendo ids\\
\verb|    |session.entropy\_length    = 0       ; N�mero de bytes a leer del fichero de entrop�a.\\
\verb|    |session.entropy\_file      =         ; El fichero que generar� la entrop�a.\\
\verb|    |session.cache\_limiter     = nocache ; Sin cache de sessiones.\\
\verb|    |session.cache\_expire      = 180     ; Tiempo de expiraci�n del documento.\\
\verb|    |session.use\_trans\_sid     = 0       ; Usar sid transportable si est� activado en la compilaci�n\\
\\
\end{longtable}
\end{center}


\subsection{Seguridad}

Php es un poderoso lenguaje de script, permite a su usuario tener bastante control sobre el sistema y a atacantes maliciosos muchas opciones de alcanzar su objetivo.

Un administrador de sistemas no debe suponer que un sistema es completamente seguro con solo tener las actualizaciones de seguridad instaladas, un sistema que muestra c�digo al exterior no es seguro, aunque el resultado sea HTML, se expone a ataques de formas muy diversas y a fallos de seguridad desconocidos.

Limitar al m�ximo el acceso que permite php es entonces una necesidad.

\subsection{Configurando safe php}

PHP tiene un modo llamado \htmladdnormallink{safe-mode}{http://www.php.net/manual/en/features.safe-mode.php} que permite una mayor seguridad, una configuraci�n para Safe mode recomendada es:\\
\\
\noindent \begin{tabular}{|l|}\hline
\verb|    |\ref{mqq} magic\_quotes\_qpc = On\\
\verb|    |\ref{auf} allow\_url\_fopen = No\\
\verb|    |\ref{rg} register\_globals = Off\\
\verb|    |\ref{sm} safe\_mode = On\\
\verb|    |\ref{smid} safe\_mode\_include\_dir = "/usr/share/gosa:/var/spool/gosa"\\
\verb|    |\ref{smed} safe\_mode\_exec\_dir = "/usr/lib/gosa"\\
\verb|    |\ref{smaev} safe\_mode\_allowed\_env\_vars = PHP\_,LANG\\
\verb|    |\ref{ob} open\_basedir = "/etc/gosa:/var/spool/gosa:/var/cache/gosa:/usr/share/gosa:/tmp"\\
\verb|    |\ref{ip} include\_path = ".:/usr/share/php:/usr/share/gosa:/var/spool/gosa:/usr/share/gosa/safe\_bin"\\
\verb|    |\ref{df} disable\_functions = system, shell\_exec, passthru, phpinfo, show\_source\\
\hline \end{tabular}
\\
En el caso de que vayamos a usar SuPHP, debemos dar los siguientes permisos al directorio /var/lib/php4:\\
\begin{tabular}{|l|}\hline 
\#chmod 1777 /var/lib/php4\\
\hline \end{tabular}

Ya que cada usuario que ejecute PHP guardara la sesi�n con ese usuario.


\section{ M�dulos de PHP necesarios}

En esta secci�n se explicaran los pasos para conseguir compilar y usar los m�dulos necesarios o importantes para GOsa, se recomienda instalar todos los m�dulos, incluso los que no son necesarios.

\subsection{ ldap.so}

M�DULO NECESARIO

\indent Este m�dulo no necesita ninguna configuraci�n especial para funcionar.

\indent Solo se conoce un problema: No puede conectarse PHP+Apache con un servidor LDAP que pida Certificado v�lido. Con lo cual la comunicaci�n ser� segura, ya que se puede usar SSL, pero no estar� garantizada.

\subsection{ mysql.so}

M�DULO OPCIONAL

\indent Este m�dulo no necesita ninguna configuraci�n especial para funcionar.

\indent Sirve para albergar configuraciones del plugin imap - sieve.

\subsection{ imap.so}

M�DULO OPCIONAL

\indent El m�dulo instalado al compilar PHP funcionara, pero tendr� una importante carencia, la funci�n getacl que da control sobre las carpetas, as� que necesitaremos un parche y una serie de pasos para compilar el m�dulo para su uso en GOsa.

Nos bajamos el parche de \htmladdnormallink{php4-imap-getacl.patch}{ftp://oss.gonicus.de/pub/gosa/patches/php4-imap-getacl.patch} y lo ponemos en /usr/src, como tenemos las fuentes de PHP en /usr/src, ejecutamos los siguientes comandos:\\
\\
\noindent \begin{tabular}{|l|}\hline
\#cd /usr/src/php4.3-XXX/extensions/imap\\
\#make clear\\
\#patch -p1 </usr/src/patch/php4-imap-getacl.patch\\
\#phpize\\
\#./configure\\
\#make\\
\#make install\\
\hline \end{tabular}


Esto configurara e instalara correctamente el m�dulo.

\subsection{ gd.so}

M�DULO OPCIONAL

\indent Este m�dulo no necesita ninguna configuraci�n especial para funcionar.

\indent El m�dulo es usado para el manejo de gr�ficos, tambien usado por el sistema de plantillas smarty.

\subsection{cups}

M�DULO OPCIONAL

\indent Para utilizar el m�dulo Cups para la selecci�n de la impresora en Posix, primero debemos descargar las fuentes de cups de \htmladdnormallink{http://www.cups.org/software.php}{http://www.cups.org/software.php} y descomprimirlas en /usr/src, ejecutamos entonces los siguientes comandos:\\
\\
\noindent \begin{tabular}{|l|}\hline
\#cd /usr/src/cups-1.1.XX/scripting/php\\
\#phpize\\
\#./configure\\
\#make\\
\#make install\\
\#echo \verb|"extension=cups.so" >>| /etc/php4/apache2/php.init\\
\#/etc/init.d/apache2 reload\\
\hline \end{tabular}


\subsection{krb}

M�DULO OPCIONAL

\indent Este m�dulo necesita tener instaladas las fuentes de kerberos del MIT, ya que no se puede compilar con las fuentes de kerberos de Heimdal.

\indent El m�dulo interactuar� con los servidores Kerberos para actualizar las claves de los usuarios creados.

Se descargara de \htmladdnormallink{PECL}{http://pecl.php.net/kadm5}, y lo descomprimiremos en /usr/src, debemos tener tambien las fuentes del kerberos del MIT, las cuales descomprimiremos en /usr/src, con ello hacemos (sustituimos X.X por las respectivas versiones de los programas):\\
\\
\noindent \begin{tabular}{|l|}\hline
\#cd /usr/src/kadm5-0.X.X/scripting/php\\
\#cp config.m4 config.m4.2\\
\#sed \verb|s/krb5-1\.2\.4\/src\/include/krb5-1\.X\.X\/src\/lib/| config.m4.2 >config.m4\\
\#rm -f config.m4.2\\
\#phpize\\
\#./configure\\
\#make\\
\#make install\\
\#echo \verb|"extension=kadm5.so" >>| /etc/php4/apache2/php.ini\\
\#/etc/init.d/apache2 reload\\
\hline \end{tabular}




\chapter{GOsa}

\chapter{Servidores de Dominios de Nombres - DNS}
\chapter{Servidores de Correo Electr�nico}
\section{Postfix}
\section{Cyrus-IMAP}
\chapter{Servidores de archivos - Samba}

\include{manual_gosa_es_printing}
\include{manual_gosa_es_proxy}
\include{manual_gosa_es_gw}
\include{manual_gosa_es_ssh}
\include{manual_gosa_es_vpn}
\include{manual_gosa_es_ftp}
\include{manual_gosa_es_im}

\chapter{Los Servidores}
\label{servidores}

\bibliography{referencias_gosa}
\bibliographystyle{unsrt}

\end{document}
