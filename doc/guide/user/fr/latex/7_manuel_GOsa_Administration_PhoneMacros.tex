%% LyX 1.3 created this file.  For more info, see http://www.lyx.org/.
%% Do not edit unless you really know what you are doing.
\documentclass[frenchb]{article}
\usepackage[T1]{fontenc}
\usepackage[latin1]{inputenc}
\usepackage{geometry}
\geometry{verbose,a4paper,tmargin=10mm,bmargin=20mm,lmargin=10mm,rmargin=10mm,headheight=10mm,headsep=10mm,footskip=10mm}
\usepackage{array}
\usepackage{color}
\usepackage{graphicx}
\usepackage{amssymb}

\makeatletter

%%%%%%%%%%%%%%%%%%%%%%%%%%%%%% LyX specific LaTeX commands.
%% Because html converters don't know tabularnewline
\providecommand{\tabularnewline}{\\}

\AtBeginDocument{
  \renewcommand{\labelitemi}{\(\centerdot\)}
  \renewcommand{\labelitemii}{\(\centerdot\)}
  \renewcommand{\labelitemiii}{\(\centerdot\)}
  \renewcommand{\labelitemiv}{\normalsize\(\centerdot\)}
}

\usepackage{babel}
\makeatother
\begin{document}

\title{\textbf{ADMINISTRATION}}

\maketitle

\section{Phone macros}

L'administrateur peut configurer les phone macros � partir du bouton
\textbf{\textcolor{blue}{Phone macros}} dans le menu de gauche dans
la partie \textbf{Administration}. La page \textsf{Phone macro management}
s'affiche. 

La page est divis�e en trois colonnes :

- la premi�re colonne est destin�e � afficher les \_,

- la deuxi�me 

- la troisi�me colonne contient les ic�nes qui sont les actions que
l'on peut executer sur ces \_.

Les fl�ches en ent�te ( \includegraphics{images/list_back.png}, \includegraphics{images/list_root.png},
\includegraphics{images/list_home.png} ) servent � modifier l'affichage
selon le d�partement. Ces ic�nes pr�dominent sur toutes autres formes
de s�l�ction d'affichage. 

\medskip{}
C'est � partir de la page \textsf{Phone macro management} que l'administrateur
g�re les phone macros instaur�es dans la soci�t�.

Il est possible de modifier l'affichage des phone macros en utilisant
le tableau intitul� Filtres \includegraphics{images/rocket.png}.

L'administrateur peut faire une recherche sur le nom :

- cliquez sur l'ast�risque (�toile : {*}) pour voir appara�tre toutes
\_;

- cliquez sur une lettre et tous les noms des \_ d�butant par cette
lettre s'afficheront;

- cliquez sur un num�ro et tous les noms des \_ d�butant par ce num�ro
s'afficheront;

- recherche rapide \includegraphics{images/search.png} : remplissez
le champ par le nom de \_ et ensuite cliquez sur le bouton Appliquer.

\medskip{}
Pour cr�er et configurer \_, l'administrateur doit cliquer sur le
bouton \includegraphics{images/list_new_macro.png}.

Appara�t l'espace de configuration.

Pour enregistrer la configuration, cliquez sur le bouton Termin�,
pour revenir � la page pr�c�dente cliquez sur le bouton Annuler.


\newpage
\subsection{Informations g�n�rales}

\begin{tabular}{|p{5cm}|p{10cm}|}
\hline 
Macro name\textcolor{red}{{*}}&
Ins�rez le nom de la macro\tabularnewline
\hline 
Afficher le nom\textcolor{red}{{*}}&
\textcolor{red}{?}\tabularnewline
\hline 
Base&
\textcolor{red}{?}\tabularnewline
\hline 
Description&
\textcolor{red}{?}\tabularnewline
\hline 
Visible for user&
\textcolor{red}{?}\tabularnewline
\hline
\end{tabular}

\begin{itemize}
\medskip{}
\item Macro text
\end{itemize}

\bigskip{}
\subsection{Parameter}

\textcolor{red}{?}


\bigskip{}
\subsection{R�f�rences}

\textcolor{red}{?}
\end{document}
